\documentclass{article}
\usepackage[shortlabels]{enumitem}
\usepackage{graphicx}
\usepackage{minted}
\graphicspath{{$HOME/texdefaults/images}}

\begin{document}

\begin{titlepage}
	\begin{center}
		\Huge
		\textbf{Lista 6}	

		\vspace{0.5cm}
		\LARGE
		
		\textbf{Caio Ian Rozendo Silva de Melo}
			
		\vfill
		\includegraphics[scale=0.8]{UFPE}
		\vfill

		Métodos Computacionais

		\vspace{0.8cm}


		\Large
		UFPE\\
		Brasil\\
		\today

	\end{center}
\end{titlepage}


\section{}
	\begin{enumerate}[a)]
		\item \mbox{}
			\begin{minted}[mathescape, linenos, frame=lines, framesep=2mm, numbersep=5pt, breaklines]{c}
// Ordenação de um vetor de inteiros por seleção, onde $r = N$
void ord_selecao(int *v, int r) {
	int i, j, s;
				
	for(i = 0; i < r; ++i) {
		for(j = 0; j < r; ++j) {
			if(v[j] < v[i]) {
				s = v[i];
				v[i] = v[j];
				v[j] = s;
			}
		}
	}
}
			\end{minted}
			Independente do vetor estar parcialmente ordenado, cada elemento faz $N^2 = 64$ comparações(podendo ser marginalmente melhorado se ``pularmos'' a comparação com o elemento do mesmo índice). Já a quantidade de trocas $T = 2$
		\item \mbox{}
			\begin{minted}[mathescape, linenos, frame=lines, framesep=2mm, numbersep=5pt, breaklines]{c}
// Ordenação de um vetor de inteiros por inserção, onde $max = N$ e $b$ é umbooleano  para garantir que apenas a primeira recursão varrerá o vetor inteiro.
void ord_ins(int *v, int r, int max, int b) {
	int s;

	if( (r - 1) >= 0 && (v[r-1] > v[r]) ) {
		s = v[r];
		v[r] = v[r-1];
		v[r-1] = s;
		++t;
		if(r+1 < max) ord_ins(v, r+1, max, 1);
	}
	if(b == 0 && r > 0) ord_ins(v, r-1, max, b);
	++c;
}
			\end{minted}
			Foram feitas $C = 8$ comparações e $T = 2$ trocas para o vetor dado.

	\end{enumerate}
\section{}
	\begin{enumerate}[a)]
		\item Inicialmente um vetor de $N = 100$ é inicializado com valores aleatórios de $[0; 10 \cdot N)$, após isso envia-se o vetor para ser ordenado or seleção, onde faz-se uma cópia na memória dos valores do vetor anterior; dentro do laço, à cada iteração, procura-se a posição no vetor de menor índice em relação ao pivô ``i'' e assim, faz-se o swap dele com o valor no pivô(mesmo que esse valor seja o pivô em si), e após isso soma-se a quantidade de comparações com o valor prèvio e repete-se até o valor $n - 1$ e o valor de comparações é igual ao total de comparações feitas. Após isso é imprimido o vetor ordenado resultante.
	Nas inserções, faz-se praticamente o mesmo que acima, porém, com menos comparações pois não é necessário verificar valores anteriores a não ser que hajam trocas de posições.
	\end{enumerate}

\end{document}
