\documentclass{article}
\usepackage[shortlabels]{enumitem}
\usepackage{graphicx}
\usepackage{minted}
\graphicspath{{$HOME/texdefaults/images}}

\begin{document}

\begin{titlepage}
	\begin{center}
		\Huge
		\textbf{Lista 6}	

		\vspace{0.5cm}
		\LARGE
		
		\textbf{Caio Ian Rozendo Silva de Melo}
			
		\vfill
		\includegraphics[scale=0.8]{UFPE}
		\vfill

		Métodos Computacionais

		\vspace{0.8cm}


		\Large
		UFPE\\
		Brasil\\
		\today

	\end{center}
\end{titlepage}


\section{}
	\begin{enumerate}[]
	\begin{minted}[mathescape, linenos, frame=lines, framesep=2mm, numbersep=5pt, breaklines]{c}
// Função para trocar os valores dados
void swapi(int *a, int *b) {
	int c = (*a);
	(*a) = (*b);
	(*b) = c;
}
	\end{minted}
	\end{enumerate}

	\begin{enumerate}[a)]
		\item \mbox{}
			\begin{minted}[mathescape, linenos, frame=lines, framesep=2mm, numbersep=5pt, breaklines]{c}
// Ordenação de um vetor de inteiros por seleção, onde $r = N$
void ord_selecao(int *v, int r) {
	int i, j, s;
				
	for(i = 0; i < r; ++i) {
		for(j = 0; j < r; ++j) {
			if(v[i] < v[j] && i < j) swapi(&v[i], &v[j]);
		}
	}
}
			\end{minted}
			Independente do vetor estar parcialmente ordenado, cada elemento faz $N^2 = 64$ comparações(podendo ser marginalmente melhorado se ``pularmos'' a comparação com o elemento do mesmo índice). Já a quantidade de trocas $T = 2$
		\item \mbox{}
			\begin{minted}[mathescape, linenos, frame=lines, framesep=2mm, numbersep=5pt, breaklines]{c}
// Ordenação de um vetor de inteiros por inserção do índice $(max,0)$, onde $max = N$ e $b$ é um booleano  para garantir que apenas a primeira recursão varrerá o vetor inteiro.
void ord_ins(int *v, int r, int max, int b) {
	int s;

	if( (r - 1) >= 0 && (v[r-1] > v[r]) ) {
		swapi(&v[r-1], &v[r]);
		if(r+1 < max) ord_ins(v, r+1, max, 1);
	}
	if(b == 0 && r > 0) ord_ins(v, r-1, max, b);
}
			\end{minted}
			Foram feitas $C = 9$ comparações e $T = 2$ trocas para o vetor dado.

	\end{enumerate}
\section{}
	\begin{enumerate}[a)]
		\item Inicialmente, um vetor de $N = 100$ é inicializado com valores aleatórios com valores compreendidos entre $[0; 10 \cdot N)$, após isso envia-se o vetor para ser ordenado por seleção, onde faz-se uma cópia na memória dos valores do vetor anterior; dentro do laço, à cada iteração, procura-se a posição no vetor no menor índice em relação ao pivô na posição ``i'' e caso ínfero ou igual, faz-se a troca entre eles; no final soma-se a quantidade de comparações com o valor prévio e repete-se até o valor $N - 1$. Após a rotina, imprime-se o vetor ordenado resultante.
		Nas inserções, faz-se ordenamento por inserções(com outra cópia do vetor inicial) que também percorre o vetor de forma linear, porém, com menos comparações pois não é necessário verificar valores anteriores a não ser que hajam trocas de posições.
	\item[d)] A quantidade de comparações feitas entre a ordenação por inserção é menor que a de seleção, porém, com algumas melhoras podemos diminuir o número de comparações na seleção, sendo no algorítmo dado igual a $T = \frac{N(N+1)}{2}$ e procurando-se pelo menor índice nos gera N trocas.
	\end{enumerate}

\end{document}
